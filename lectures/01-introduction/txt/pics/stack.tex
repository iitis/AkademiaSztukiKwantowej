\documentclass[varwidth=true,preview]{standalone}
% \usepackage{fontawesome}
\usepackage{tikz}
% \usepackage{circuitikz}
\usepackage{qcircuit}
\usetikzlibrary{shadows,arrows.meta,positioning,backgrounds,fit,chains,scopes}

\begin{document}

\tikzset{%
	materia/.style={draw, fill=blue!20, text width=6.0em, text centered, minimum height=1.5em,drop shadow},
	etape/.style={materia, text width=8em, minimum width=10em, minimum height=1em, rounded corners, drop shadow},
	linepart/.style={draw, thick, color=black!50, -LaTeX, dashed},
	line/.style={draw, thick, color=black!50, -LaTeX},
	ur/.style={draw, text centered, minimum height=0.01em},
	back group/.style={fill=yellow!20,rounded corners, draw=black!50, dashed, inner xsep=15pt, inner ysep=10pt},
  }
		\begin{tikzpicture}
			[
			  start chain=p going below,
			  every on chain/.append style={etape},
			  every join/.append style={line},
			  node distance=1 and -.15,
			]
			{
			  \node [on chain, join] {Algorithm};
			  \node [on chain, join] {Implementation};
			  \node [on chain, join] {Quantum assembler};
			  \node [on chain, join] {Control program};
			  \node [on chain, join] {Physical system};
			}
		  \end{tikzpicture}				  
\end{document}