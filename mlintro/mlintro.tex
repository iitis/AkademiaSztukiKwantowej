% !TeX spellcheck = pl_PL-Polish
\documentclass[10pt]{article}
\usepackage[utf8]{inputenc}
\usepackage[T1]{fontenc}
\usepackage{times}
\usepackage{polski}
\usepackage{indentfirst}
\usepackage[a4paper]{geometry}
\usepackage{todonotes}
\presetkeys{todonotes}{color=black, backgroundcolor=white,size=\footnotesize}{}
\usepackage{hyperref}
\usepackage{graphicx}
\usepackage{ctable}
\usepackage{subcaption}
\usepackage{float}
\usepackage{tikz}
\usepackage{amsmath}
\usepackage{amsfonts}
\usepackage{amssymb}
\usepackage{soulutf8}
\usetikzlibrary{fit,chains}

\newcommand{\todoi}[2]{\todo[inline,color=#2!30]{#1}}
\newcommand{\todoii}[3]{\todo[inline,color=#2!30,inlinewidth=#3cm,noinlinepar]{#1}}
\newcommand{\todox}{\todoii{do uzupełnienia}{red}{3}}
\DeclareRobustCommand{\hlcyan}[1]{{\sethlcolor{cyan}\hl{#1}}}
\DeclareRobustCommand{\hlgray}[1]{{\sethlcolor{lightgray}\hl{#1}}}

\begin{document}

\title{Podstawy uczenia maszynowego. Zagadnienia wprowadzające do tematyki Akademii Sztuki Kwantowej}
\author{Plan zajęć}
\maketitle

\section*{Część teoretyczna/wykładowa}

\subsection*{Wprowadzenie}

\begin{enumerate}
	\item Podstawowe zagadnienia uczenia maszynowego. Cel, problemy, ograniczenia. Jakie zadania realizujemy, jakich nie realizujemy za pomocą ML.
	\item Przykładowy problem -- klasyfikacja hiperspektralna, analiza kryminalistyczna, ślady krwi. Rola wiedzy problemowej/domenowej (lub jej braku) w zadaniu. Oczekiwany wynik
	\item Podstawowe pojęcia -- model, trening, ocena wydajności, dobór hiperparametrów, klasyfikacja/regresja, [i inne].

\end{enumerate}

\subsection*{Sztuczne sieci neuronowe}

\begin{enumerate}
	\item Wprowadzenie ,,ideowe'', omówienie przykładu ,,owoce''
	\item Budowa sieci MLP, schemat klasyfikacji dla problemu hiperspektralnego śladów krwi
	\item Analiza działania wytrenowanej sieci dla problemu. Wizualizacja wyników. Wagi sieci. Obserwacja działania
	\item Proces uczenia -- stan początkowy, propagacja wsteczna, parametry, analiza przykładowego przebiegu uczenia
	\item Proces uczenia (2) -- nadzór, rola hiperparametrów itd. Przykład zmiany architektury
	\item Optymalizacja -- porównanie architektur, ocena wydajności, wybór modelu, weryfikacja, walidacja krzyżowa, optymalizacja hiperparametrów, przetrenowanie
\end{enumerate}

\subsection*{Metody jądrowe}

\begin{enumerate}
	\item Klasyfikacja jeszcze raz, idea używania przykładów i miary podobieństwa
	\item Ograniczenia sieci neuronowych, różne sytuacje klasyfikacji wymagają różnych metod, wyjaśnialność 
	\item PCA jako narzędzie wizualizacji, macierz kowariancji
	\item SVM, optymalizacja marginesu, parametr C
	\item kernel-SVM, kernel trick, Gaussian kernel, gamma 
	\item kernel-PCA jako uogólnienie
	\item analiza, przykłady, zastosowania
\end{enumerate}

\subsection*{Sieci głębokiego uczenia}

\begin{enumerate}
	\item Problemy przetwarzania obrazów i tekstu -- ograniczenia sieci MLP
	\item Sieci złożone, głębokie hierarchie cech 
	\item Konwolucja i pooling -- model HMAX, potem rozwinięcia 
	\item Sieć konwolucyjna, analiza działania
	\item (*) Problem zależności w sekwencjach symboli przy przetwarzaniu tekstu
	\item (*) Idea sieci transformers
	\item Podsumowanie -- złożone architektury, złożone sieci
\end{enumerate}

(Dla (*) jeszcze się zastanawiam, czy i w jakiej formie to włączyć -- zobaczę ile czasu mi wyjdzie na wcześniejsze)

\end{document}